\chapter{Tutorial 3: Extraction data from Front Arena to Excel - Correlation Extraction}

Extraction of data from Front Arena to Excel is now directly possible. 


\section{Fron Arena connection session}

Before getting data, we need to connect to Front Arena. As Front Arena does not accept multiple connection of the same user we need to create a singleton class. That is the aim of the class FrontArenaSession:

\begin{verbatim}

class FrontArenaSession:
    
    class __Impl:
        
        def __init__(self, server, userid, password):
            import sys
            sys.path.insert(0, r'C:\Program Files\Front\Front Arena42\Prime\4.2')
            self.ael = __import__('ael')
            self.acm = __import__('acm')
            self.ael.connect(server, userid, password)
 
        def __del__(self):
            self.ael.disconnect()
            
        def today(self):
            return self.ael.date_today()
        
        def excel_date(self, ael_date):
            return self.ael.date_from_string('1899-12-30').days_between(ael_date)
        
        def ael_date(self, excel_date):
            return self.ael.date_from_string('1899-12-30').add_days(excel_date)
                        
    __instance = {}
 
    def __init__(self, server, userid, password):
        self._key = (server, userid, password)
        if not FrontArenaSession.__instance.has_key(self._key):
            FrontArenaSession.__instance[self._key] = FrontArenaSession.__Impl(server, userid, password)
    
    def __getattr__(self, name):
        if not name == '_key':
            return getattr(self.__instance[self._key], name)
 
    def __setattr__(self, aAttr, aValue):
        if not aAttr == '_key':
            return setattr(self.__instance[self._key], aAttr, aValue)
            
\end{verbatim} 

\section{A data manager class}

\begin{verbatim}
class DataManager:

    def __init__(self, session):
        self._session = session

    def getCorrelation(self, name):
        corr = self._session.ael.CorrelationMatrix[name].correlations()
        index={}
        pairs={}
        for member in corr:   
            first = self._session.ael.Instrument[member.recaddr0].insid
            second = self._session.ael.Instrument[member.recaddr1].insid
            pairs[(first,second)] = member.corr    
            index[first] = None
            index[second] = None
        index = index.keys()
        index.sort()
        return CorrelationMatrix(index, pairs, name)
\end{verbatim}


\section{A correlation class}

\begin{verbatim}
class Matrix:

    def __init__(self, rows, cols):
        self._rows = rows
        self._cols = cols
        self._datas = range(0, rows*cols)

    def __getitem__(self, index):
        return self._datas.__getitem__(index)

    def __setitem__(self, index, value):
        self._datas.__setitem__(index, value)

    def __len__(self):
        return len(self._datas)

    def size1(self):
        return self._rows
        
    def size2(self):
        return self._cols

    def operator(x, y):
        if (x > self._rows) or (y > self._cols) or (x < 0) or (y < 0):
            raise Exception , "bad value for index"
        return self._datas[x*self._cols+y]


class CorrelationMatrix:

    def __init__(self, instruments, corrPairs, name):
        self.instruments = instruments
        self._corrPairs = corrPairs
        self.matrix = Matrix(len(instruments), len(instruments))
        self.name = name
        for x in instruments:
            self._corrPairs[(x, x)] = 1.0
        offset = 0
        for x in instruments:
            for y in instruments:
                val = 0.0
                try:
                    val = self._corrPairs[(x, y)] 
                except:
                    val = self._corrPairs[(y, x)]
                self.matrix.__setitem__(offset, val) 
                offset += 1
        
    def correlation(self, inst1, inst2):
        try:
            return self._corrPairs[(inst1, inst2)]
        except:
            return self._corrPairs[(inst2, inst1)]
          
\end{verbatim}

\section{Putting all together} 

Go to see the file correlationtutorial.xls to see the result.